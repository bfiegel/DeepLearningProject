\documentclass{article}


\usepackage{arxiv}
\usepackage[utf8]{inputenc} % allow utf-8 input
\usepackage[T1]{fontenc}    % use 8-bit T1 fonts
\usepackage{hyperref}       % hyperlinks
\usepackage{url}            % simple URL typesetting
\usepackage{booktabs}       % professional-quality tables
\usepackage{amsfonts}       % blackboard math symbols
\usepackage{nicefrac}       % compact symbols for 1/2, etc.
\usepackage{microtype}      % microtypography
\usepackage{lipsum}

\title{Deep Learning Project: GAN Artwork}


\author{
	Kevin Steele \\
	Department of Electrical Engineering \\
	University of Iowa\\
	\texttt{kevin-steele@uiowa.edu} \\
	%% examples of more authors
	\And
	Brain Fiegel \\
	Department of Electrical Engineering \\
	University of Iowa\\
	\texttt{brian-fiegel@uiowa.edu} \\
	\And
	Stjepan Fiolic \\
	Department of Electrical Engineering \\
	University of Iowa\\
	\texttt{stjepan-fiolic@uiowa.edu} \\
}

\begin{document}
	\maketitle
	
	\begin{abstract}
		https://www.overleaf.com/latex/templates/style-and-template-for-preprints-arxiv-bio-arxiv/fxsnsrzpnvwc

	\end{abstract}
	
	
	% keywords can be removed
	\keywords{First keyword \and Second keyword \and More}
	
	
	\section{Introduction}
	\begin{itemize}
		\item What's the goal? State your problem clear and succinct. 
		\subitem This project will create artworks using the DNA sequence of a person as a seed. A generative adversarial network (GAN) will be trained on images of landscape paintings and be tuned to produce a customized artwork for each different DNA sequence. If successful, the algorithm will be deployed at the Iowa Neuroscience Institute to create souvenirs for research subjects.  
		\item What are the independent variables (inputs) and dependent variables (outputs)? 
		\subitem Input = DNA sequence of a research subject | Output = artwork

	\end{itemize}

	
	
	\section{Related Works (Optional)}
	\label{sec:headings}
	
	\begin{itemize}
		\item If you find it necessary to have a separate section to deep dive into other relevant works in literature, create a section titled "Related Work".
		\item Try to categorize your/other people's previous works and build a *story*. Don't just list some facts. The worst thing to do is something like: "Smith et al. did this. Johnson et al. did that. Xu et al. did this. ..." ----a simple listing of papers. Instead, tell a story (like a fun history book).
		\item For the scope of this project, it is not always necessary.
		\item If you plan to transform this course project into a journal/conference submission, it should be helpful to have it written up.
	\end{itemize}
	
	See Section \ref{sec:headings}.
	
	
	\section{Problem Definition}
	\label{sec:others}
	\lipsum[8] \cite{kour2014real,kour2014fast} and see \cite{hadash2018estimate}.
	
	The documentation for \verb+natbib+ may be found at
	\begin{center}
		\url{http://mirrors.ctan.org/macros/latex/contrib/natbib/natnotes.pdf}
	\end{center}
	Of note is the command \verb+\citet+, which produces citations
	appropriate for use in inline text.  For example,
	\begin{verbatim}
	\citet{hasselmo} investigated\dots
	\end{verbatim}
	produces
	\begin{quote}
		Hasselmo, et al.\ (1995) investigated\dots
	\end{quote}
	
	\begin{center}
		\url{https://www.ctan.org/pkg/booktabs}
	\end{center}
	
	
	\subsection{Figures}
	\lipsum[10] 
	See Figure \ref{fig:fig1}. Here is how you add footnotes. \footnote{Sample of the first footnote.}
	\lipsum[11] 
	
	\begin{figure}
		\centering
		\fbox{\rule[-.5cm]{4cm}{4cm} \rule[-.5cm]{4cm}{0cm}}
		\caption{Sample figure caption.}
		\label{fig:fig1}
	\end{figure}
	
	\subsection{Tables}
	\lipsum[12]
	See awesome Table~\ref{tab:table}.
	
	\begin{table}
		\caption{Sample table title}
		\centering
		\begin{tabular}{lll}
			\toprule
			\multicolumn{2}{c}{Part}                   \\
			\cmidrule(r){1-2}
			Name     & Description     & Size ($\mu$m) \\
			\midrule
			Dendrite & Input terminal  & $\sim$100     \\
			Axon     & Output terminal & $\sim$10      \\
			Soma     & Cell body       & up to $10^6$  \\
			\bottomrule
		\end{tabular}
		\label{tab:table}
	\end{table}
	
	\subsection{Lists}
	\begin{itemize}
		\item Lorem ipsum dolor sit amet
		\item consectetur adipiscing elit. 
		\item Aliquam dignissim blandit est, in dictum tortor gravida eget. In ac rutrum magna.
	\end{itemize}
	
	
	\section{Data}
	
	
	\bibliographystyle{unsrt}  
	%\bibliography{references}  %%% Remove comment to use the external .bib file (using bibtex).
	%%% and comment out the ``thebibliography'' section.
	
	
	%%% Comment out this section when you \bibliography{references} is enabled.
	\begin{thebibliography}{1}
		
		\bibitem{kour2014real}
		George Kour and Raid Saabne.
		\newblock Real-time segmentation of on-line handwritten arabic script.
		\newblock In {\em Frontiers in Handwriting Recognition (ICFHR), 2014 14th
			International Conference on}, pages 417--422. IEEE, 2014.
		
		\bibitem{kour2014fast}
		George Kour and Raid Saabne.
		\newblock Fast classification of handwritten on-line arabic characters.
		\newblock In {\em Soft Computing and Pattern Recognition (SoCPaR), 2014 6th
			International Conference of}, pages 312--318. IEEE, 2014.
		
		\bibitem{hadash2018estimate}
		Guy Hadash, Einat Kermany, Boaz Carmeli, Ofer Lavi, George Kour, and Alon
		Jacovi.
		\newblock Estimate and replace: A novel approach to integrating deep neural
		networks with existing applications.
		\newblock {\em arXiv preprint arXiv:1804.09028}, 2018.
		
	\end{thebibliography}
	
	
\end{document}